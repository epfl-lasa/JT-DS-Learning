\documentclass{article}

\usepackage{color}
\title{A Small \LaTeX{} Article Template\thanks{To your mother}}
\author{Your Name  \\
	Your Company / University  \\
	\and 
	The Other Dude \\
	His Company / University \\
	}

\date{\today}
% Hint: \title{what ever}, \author{who care} and \date{when ever} could stand 
% before or after the \begin{document} command 
% BUT the \maketitle command MUST come AFTER the \begin{document} command! 
\begin{document}
\vspace{-10pt}
In response to \underline{Reviewer: 5 (Review30371)} (Comments 1-3). Following we mathematically and graphically analyze the proposed local behavior synergy extraction scheme which is composed of dimensionality reduction of the joint trajectories (via. PCA or K-PCA) and grouping of local behavior regions with GMM on this lower dimensional space. The $K$ clusters found in this embedded space correspond to the number of $K$ synergy matrices that will parametrize $\mathcal{A}(q)$ and analyze the feasibility of applying F-PCA to this scheme. Let's begin by analyzing the proposed DS-based control-law:
\begin{equation}
\dot{q} = -\mathcal{A}(q)J^{T}(q)(H(q)-x^*)
\end{equation}
The intuition behind this control-law is that $\mathcal{A}(q) = \sum\limits_{k=1}^{K}\theta'(\phi(q))A_k$ is a linear combination of time-invariant linear matrices $A_k$, these matrices are the ``local behavior synergy matrices" that shape the motion in joint-space. Meaning that, given the joint-space velocity vector representing the task-space error $J^{T}(q)(H(q)-x^*)$, the resulting motion is biased to use a particular set of joints, defined in each $A_k$. Each $A_k$ is thus activated depending on the current ``joint-posture" $q$ (represented in the lower-dimensional space $\phi(q)$) via the scheduling/activation function $\theta'(\phi(q))$. This is our proposed definition of ``local behavior synergies", which is not equivalent to the notion of ``hand posture synergies" as suggested by the reviewer. The fact that we use PCA or K-PCA to represent to activation function is to alleviate the complexity of extracting the local synergy regions in joint-space. From the classical definition of joint synergies, when transforming the joint trajectories via PCA/K-PCA to a lower dimensional-space we are indeed working in the ``joint postural synergy" space as the reviewer suggests. But, solely the activation function $\theta'(\phi(q))$ is represented by these joint postural synergies. The $A_k$ matrices themselves are the ``local behavior synergies" and these live in joint-space $A_k\in \mathbb{R}^{m \times m}$ and are time-invariant. Note that, if we were to bypass the DR step our state-dependent system matrix would be the following $\mathcal{A}(q)= \sum\limits_{k=1}^{K}\theta(q)A_k$, which through our definition is still a combination of ``local behavior synergy matrices". As shown in TABLE I of the manuscript and discussed in Section V, by using a lower-dimensional manifold to represent the joint trajectories, we are getting rid of outliers, noise and redundancies that might arise from the raw joint demonstrations. Hence, through DR we are capable of robustly extracting the local behavior synergies from raw demonstrations. Following we demonstrate this on the pouring behavior dataset.

In Fig.




This activation is defined by  which is the posterior probability of a GMM that models these local linear motions with. The idea of using a DR scheme to parametrize this activation function stems from the fact that



%\begin{thebibliography}{9}
%\bibitem[Doe]{doe} \emph{First and last \LaTeX{} example.},
%John Doe 50 B.C. 
%\end{thebibliography}

\end{document}





\end{newlfm}
\end{document}